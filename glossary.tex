\makeglossaries
\newglossaryentry{Reviewer}{
name = Reviewer,
description ={Benutzerkonto, das nur Leserechte hat und auf Ergebnisse zugreifen kann  aber keine Workflows erstellen oder ändern kann},
plural = Reviewer
}

\newglossaryentry{Apache Airflow}{
name = Apache Airflow,
description ={open-source Workflow Management Plattform für Workflows}
}

\newglossaryentry{Nutzer}{
name = Developer,
description ={Benutzerkonto, das Workflows erstellen und ausführen kann},
plural = Developer
}

\newglossaryentry{Workflow}{
name = Workflow-Instanz,
description = {Gerichteter Graph von Aufgaben mit Parametern},
plural = Workflow-Instanzen
}

\newglossaryentry{Admin}{
name = Admin,
description ={(Auch Administrator) Benutzerkonto, das die Konfiguration des Servers ändern kann und Benutzer verwalten kann},
plural = Administratoren
}

\newglossaryentry{Client-Server}{
name = Client-Server,
description = {Eine Software Architektur, bei der Anwender (Clients) Dienste von einem Server anfordern können. Der Server bedient dabei mehrere Clients},
plural = Client-Server
}

\newglossaryentry{Machine Learning}{
name = Machine Learning,
description = {Anwendung der künstlichen Intelligenz, die aus Daten selbstständig versucht, Muster und Zusammenhänge zu lernen und sich darauf basierend, selbstständig zu verbessern},
plural = Machine Learnings
}

\newglossaryentry{SQL}{
name = SQL,
description = {Sprache zur Abfrage, Definition und Manipulation von Daten in Datenbanken},
plural = SQLs
}

\newglossaryentry{Injektion}{
name = Injektion,
description = {Angriff auf eine Datenbank, bei dem der Angreifer versucht eigene Befehle in die Datenbank einzuschleusen},
plural = Injektionen
}

\newglossaryentry{Plotskript}{
name = Plotskript,
description = {Programm zum Erstellen von Visualisierungen von Daten},
plural = Plotskripte
}

\newglossaryentry{Key-Value-Pair}{
name = Key-Value-Pair,
description = {Paar aus einem Schlüssel und einem Wert},
}

\newglossaryentry{Python}{
name = Python,
description = {Universelle, höhere Programmiersprache},
}

\newglossaryentry{Virtual Machine}{
name = Virtual Machine,
description = {Kapselung eines Rechnersystems innerhalb eines reell existierenden Rechners, das die Architektur eines anderen Rechners nachbildet},
}

\newglossaryentry{TGDS}{
name = TGDS,
description = {Theorie Guided Data Science: Erweitert Anwendungen künstlicher Intelligenz um theoretische Modelle}
}

\newglossaryentry{M++}{
name = M++,
description = {Framework der Fakultät für Mathematik um Differenzialgleichungen nach der Finite Elemente Methode zu lösen}
}

\newglossaryentry{Gruppe}{
name = Gruppe,
description = {Klassen von Benutzern mit gleichen Rechten},
plural = Gruppen
}

\newglossaryentry{GUI}{
name = GUI,
description = {Graphische Benutzeroberfläche}
}

 \newglossaryentry{ABAQUS}{
 name = ABAQUS,
 description = {Ein kommerzielles Programmpaket, mit dem sich Probleme der Festkörper-Statik und -Dynamik, der Wärmeleitung, des Elektromagnetismus und der Fluiddynamik bearbeiten lassen},
 }
 
 \newglossaryentry{benutzerdefinierte Schritte}{
 name = benutzerdefinierte Schritte,
 description = {Vom Nutzer erstelle Schritte},
 plural = benutzerdefinierten Schritte
 }
 
 \newglossaryentry{Limits}{
 name = Resourcenlimits,
 description = {Limitierungen auf die Ressourcen, die ein Workflow nutzen darf, wie Laufzeit und Festplattenspeicher},
 }
 
 \newglossaryentry{Ressourcen}{
 name = Ressourcen,
 description = {Ressourcen des Systems, beinhaltet unter anderem die CPU-Anzahl, die Speichergröße},
 }
 
 \newglossaryentry{Workflow Schablone}{
 name = Workflow,
 description = {Unparametrisierte Workflowschritte. Bei Parameterbelegung wird eine neue Workflow Instanz erstellt},
 plural = Workflows
 }
 