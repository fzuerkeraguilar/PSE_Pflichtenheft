\chapter{Testfallszenarien}

\section{Testfälle}
Folgende Testfälle testen die Musskriterien:
\renewcommand{\labelenumi}{/T\arabic{enumi}0/}
\begin{enumerate}
    \item Workflows anzeigen
    \\ \textbf{Stand:} Der \Gls{Nutzer} ist angemeldet
    \\ \textbf{Aktion:} Direkt nach der Anmeldung keine Aktion notwendig. Sonst klickt der \Gls{Nutzer} auf \enquote{Workflowübersicht}
    \\ \textbf{Reaktion:}Die Workflowübersicht wird angezeigt
    \item Workflow anlegen
    \\ \textbf{Stand:} Geöffnete Workflowübersicht
    \\ \textbf{Aktion:} Der \Gls{Nutzer} klickt auf den \enquote{Neuen Workflow erstellen}-Knopf
    \\ \textbf{Reaktion:} Der \Gls{Nutzer} wird auf die Workflowerstellungsseite weitergeleitet
    \item Unberechtigtes Workflow anlegen
    \\ \textbf{Stand:} Geöffnete Workflowübersicht
    \\ \textbf{Aktion:} Ein \Gls{Reviewer} klickt auf den \enquote{Neuen Workflow erstellen}-Knopf
    \\ \textbf{Reaktion:} Eine Nachricht über das Fehlen von Editierrechten wird angezeigt und der \Gls{Reviewer} wird nicht weitergeleitet
    \item Workflow exportieren
    \\ \textbf{Stand:} Ausgewählter Workflow wird angezeigt
    \\ \textbf{Aktion:} Der \gls{Nutzer} klickt auf den \enquote{Workflow exportieren}-Knopf
    \\ \textbf{Reaktion:} Eine .yml oder .json Datei wird heruntergeladen
    \item Workflowinformationen einsehen
    \\ \textbf{Stand:} Geöffnete Workflowübersicht
    \\ \textbf{Aktion:} Der \gls{Nutzer} klickt auf den \enquote{Workflow Information}-Knopf
    \\ \textbf{Reaktion:} Eine Übersicht über die Attribute des Workflows wird angezeigt
    \item Workflow umbenennen
    \\ \textbf{Stand:} Geöffnete Workflowübersicht
    \\ \textbf{Aktion:} Der \gls{Nutzer} klickt auf den \enquote{Workflow Information}-Knopf und ändert unter \enquote{Workflow name} den Namen des Workflows
    \\ \textbf{Reaktion:} Der Name des Workflows wird geändert
    \item Workflow importieren
    \\ \textbf{Stand:} Der \gls{Nutzer} ist auf der Workflowerstellungsseite 
    \\ \textbf{Aktion:} Der \gls{Nutzer} lädt eine gültige .yml oder .json Datei hoch 
    \\ \textbf{Reaktion:} Der Workflow wird zur Liste der Workflows hinzugefügt
    \item Konfiguration eines Workflows anpassen
    \\ \textbf{Stand:} Der \gls{Nutzer} ist auf der Workflowbearbeitungsseite
    \\ \textbf{Aktion:} Der \gls{Nutzer} klickt wählt eine config-Datei aus und bearbeitet einen der Parameter im Texteditor
    \\ \textbf{Reaktion:} Der neue Parameter für den ausgewählten Workflow wird übernommen
    \item Anmelden
    \\ \textbf{Stand:} Login-Seite wird angezeigt
    \\ \textbf{Aktion:} Der \gls{Nutzer} trägt seine korrekte Daten ein und bestätigt sie
    \\ \textbf{Reaktion:} Der \gls{Nutzer} wird auf die Startseite gebracht
    \item Workflow starten
    \\ \textbf{Stand:} Geöffnete Workflowübersicht 
    \\ \textbf{Aktion:} \enquote{Starten}-Knopf eines Workflows wird gedrückt
    \\ \textbf{Reaktion:} Workflow wird gestartet
    \item Workflow abbrechen
    \\ \textbf{Stand:} Geöffnete Workflowübersicht
    \\ \textbf{Aktion:} \enquote{Stop}-Knopf eines Workflows wird gedrückt
    \\ \textbf{Reaktion:} Ausführung des Workflows wird unterbrochen
    \item Workflow löschen
    \\ \textbf{Stand:} Geöffnete Workflowübersicht
    \\ \textbf{Aktion:} \enquote{Löschen}-Knopf eines Workflows wird gedrückt
    \\ \textbf{Reaktion:} Workflow wird gelöscht und aus der Liste der Workflows gelöscht
    \item Benachrichtigen bei Fehlschlag eines Workflows
    \\ \textbf{Stand:} Gestarteter Workflow schlägt auf Grund eines Fehlers fehl
    \\ \textbf{Aktion:} Keine Aktion vom \gls{Nutzer} notwendig
    \\ \textbf{Reaktion:} Der \gls{Nutzer} der den Workflow gestartet hat bekommt eine Benachrichtigung angezeigt, dass der gestartete Workflow fehlgeschlagen ist
    \item Anmelden
    \\ \textbf{Stand:} Login-Seite
    \\ \textbf{Aktion:} Benutzername und Passwort eingeben
    \\ \textbf{Reaktion:} Benutzername und Passwort werden überprüft und bei Korrektheit wird man auf die Startseite weitergeleitet
    \item Registrieren
    \\ \textbf{Stand:} Registrierungsseite
    \\ \textbf{Aktion:} Registrierungsformular wird ausgefüllt
    \\ \textbf{Reaktion:} neues Benutzerkonto wird angelegt und man wird auf die Login-Seite weitergeleitet
    \item Passwort zurücksetzen
    \\ \textbf{Stand:} geöffnete Nutzerverwaltung eines Benutzerkontos
    \\ \textbf{Aktion:} \enquote{Passwort zurücksetzten}-Knopf wird gedrückt und neues Passwort wird eingegeben
    \\ \textbf{Reaktion:} ausgewähltes Benutzerkonto erhält neues Passwort
    \item Nutzerrechte anpassen
    \\ \textbf{Stand:} geöffnete Nutzerverwaltung eines Benutzerkontos
    \\ \textbf{Aktion:} \enquote{Nutzerrechte anpassen}-Knopf wird gedrückt und das Benutzerkonto wird auf den Status eines \Gls{Admin} gesetzt
    \\ \textbf{Reaktion:} ausgewähltes Benutzerkonto erhält den \Gls{Admin}-Status
    \item Account löschen
    \\ \textbf{Stand:} geöffnete Nutzerverwaltung eines Benutzerkontos
    \\ \textbf{Aktion:} \enquote{Account löschen}-Knopf wird gedrückt
    \\ \textbf{Reaktion:} ausgewähltes Benutzerkonto mit seinen Daten wird gelöscht
    \item Account sperren
    \\ \textbf{Stand:} geöffnete Nutzerverwaltung eines Benutzerkontos
    \\ \textbf{Aktion:} \enquote{Account sperren}-Knopf wird gedrückt
    \\ \textbf{Reaktion:} ausgewähltes Benutzerkonto wird vom einloggen abgehalten
    \item Account entsperren
    \\ \textbf{Stand:} geöffnete Nutzerverwaltung eines gesprerrten Benutzerkontos
    \\ \textbf{Aktion:} \enquote{Account entsperren}-Knopf wird gedrückt
    \\ \textbf{Reaktion:} ausgewähltes Benutzerkonto kann sich erneut anmelden
\end{enumerate}


\section{Testszenarien}

\subsection*{Workflow erstellen und ausführen}
\begin{itemize}
    \item Ein \Gls{Nutzer} meldet sich an und bekommt die Startseite angezeigt
    \item Der \Gls{Nutzer} navigiert zur Workflowübersicht und bekommt alle Workflows angezeigt
    \item Der \gls{Nutzer} klickt auf den \enquote{Neuen Workflow erstellen}-Knopf und bekommt den \enquote{Worflow erstellen} Dialog angezeigt
\end{itemize}

%
\subsection*{Workflow überprüfen}
\begin{itemize}
    \item Ein \Gls{Reviewer} meldet sich an und bekommt die Startseite angezeigt
    \item Der \Gls{Reviewer} navigiert zur Workflowübersicht und bekommt alle Workflows angezeigt
    \item Der \Gls{Reviewer} wählt einen laufenden Workflows aus und bekommt den Status des Workflows und ausgewählte Kennwerte angezeigt
    \item Der \gls{Reviewer} meldet sich ab
\end{itemize}

\subsection*{Workflow exportieren und importieren}
\begin{itemize}
    \item Ein \gls{Nutzer} meldet sich an und bekommt die Startseite angezeigt
    \item Der \gls{Nutzer} navigiert zur Workflowübersicht und bekommt alle Workflows angezeigt
    \item Der \gls{Nutzer} wählt eines der Workflows an und klickt auf den \enquote{Exportieren}-Knopf und eine .yml oder .json-Datei wird heruntergeladen
    \item Der \gls{Nutzer} klickt auf den \enquote{Neuen Workflow erstellen}-Knopf und bekommt den \enquote{Worflow erstellen} Dialog angezeigt
    \item Der \gls{Nutzer} lädt eine Workflow-Konfiguration hoch und der neue Workflow wird erstellt
    \item Der \gls{Nutzer} meldet sich ab
\end{itemize}


\subsection*{Ergebnisse eines Workflows einsehen}
\begin{itemize}
    \item Ein \gls{Nutzer} meldet sich an und bekommt die Startseite angezeigt
    \item Der \gls{Nutzer} navigiert zur Workflowübersicht und bekommt alle Workflows angezeigt
    \item Der \gls{Nutzer} wählt einen abgeschlossenen Workflow aus und bekommt eine Übersicht des ausgewählten Workflows angezeigt
    \item Der \gls{Nutzer} lädt sich die Ergebnisse des Workflows herunter
    \item Der \gls{Nutzer} meldet sich ab
\end{itemize}



\subsection*{Registrierung}
\begin{itemize}
    \item Ein neuer Benutzer ruft die Weboberfläche auf und bekommt die Login-Seite angezeigt
    \item Der neue Benutzer navigiert zur Registrierung
    \item Der neue Benutzer füllt das Registrierungsformular mit seinen Daten aus und gibt sein Passwort ein
    \item Der neue Benutzer schickt das Formular ab und gelangt zurück auf die Login-Seite
    \item Der neue Benutzer wartet auf die Freischaltung des neuen Kontos durch einen Administrator
    \item Der neue Benutzer meldet sich mit seinem neuem \Gls{Reviewer}konto an
\end{itemize}

\subsection*{Nutzerverwaltung}
\begin{itemize}
    \item Ein \Gls{Admin} meldet sich an und bekommt die Startseite angezeigt
    \item Der \Gls{Admin} navigiert zur Benutzerverwaltung
    \item Der \Gls{Admin} bekommt alle Nutzerkonten und ihre Rollen angezeigt
    \item Der \Gls{Admin} wählt ein Konto aus und ändert seine Rolle von \Gls{Reviewer} zu \Gls{Admin}
    \item Der \Gls{Admin} wählt ein anderes Konto aus und setzt sein Passwort zurück
    \item Der \Gls{Admin} wählt einen noch nicht freigeschalteten Nutzer aus und schaltet ihn frei
    \item Der \Gls{Admin} meldet sich ab
\end{itemize}


\subsection*{Hinzufügen von Computation Targets}
\begin{itemize}
    \item Ein \Gls{Admin} meldet sich an und ruft die Startseite auf
    \item Der \Gls{Admin} navigiert zur Serververwaltung
    \item Der \gls{Admin} erhöht den verfügbaren Hauptspeicher für die Workflows
    \item Der \gls{Admin} meldet sich ab
\end{itemize}
