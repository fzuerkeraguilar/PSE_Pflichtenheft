\chapter{Testfallszenarien}

\section{Testfälle}
Folgende Testfälle testen die Musskriterien:
\renewcommand{\labelenumi}{/T\arabic{enumi}0/}
\begin{enumerate}
    \item Workflow anlegen
    \\ \textbf{Stand:} Geöffnete Workflowübersicht
    \\ \textbf{Aktion:} Der \Gls{Nutzer} klickt auf den \enquote{Neuen Workflow erstellen}-Knopf
    \\ \textbf{Reaktion:} Der \Gls{Nutzer} wird auf die Workflowerstellungsseite weitergeleitet
    \item Workflow exportieren 
    \\ \textbf{Stand:} Ausgewählter Workflow wird angezeigt
    \\ \textbf{Aktion:} Der \gls{Nutzer} klickt auf den \enquote{Workflow exportieren}-Knopf
    \\ \textbf{Reaktion:} Eine .yml oder .json Datei wird heruntergeladen
    \item Workflow importieren 
    \\ \textbf{Stand:} Der \gls{Nutzer} ist auf der Workflowerstellungsseite 
    \\ \textbf{Aktion:} Der \gls{Nutzer} lädt eine gültige .yml oder .json Datei hoch 
    \\ \textbf{Reaktion:} Der Workflow wird zur Liste der Workflows hinzugefügt
    \item Anmelden
    \\ \textbf{Stand:} Login-Seite wird angezeigt
    \\ \textbf{Aktion:} Der \gls{Nutzer} trägt seine korrekte Daten ein und bestätigt sie
    \\ \textbf{Reaktion:} Der \gls{Nutzer} wird auf die Startseite gebracht
    \item Workflow starten
    \\ \textbf{Stand:} Geöffnete Workflowübersicht 
    \\ \textbf{Aktion:} \enquote{Starten}-Knopf eines Workflows wird gedrückt
    \\ \textbf{Reaktion:} Workflow wird gestartet
    \item Workflow abbrechen
    \\ \textbf{Stand:} Geöffnete Workflowübersicht
    \\ \textbf{Aktion:} \enquote{Stop}-Knopf eines Workflows wird gedrückt
    \\ \textbf{Reaktion:} Ausführung des Workflows wird unterbrochen
    \item Workflow löschen
    \\ \textbf{Stand:} Geöffnete Workflowübersicht
    \\ \textbf{Aktion:} \enquote{Löschen}-Knopf eines Workflows wird gedrückt
    \\ \textbf{Reaktion:} Workflow wird gelöscht und aus der Liste der Workflows gelöscht
\end{enumerate}


\section{Testszenarien}

\subsection*{Workflow erstellen und ausführen}
\begin{itemize}
    \item Ein \Gls{Nutzer} meldet sich an und bekommt die Startseite angezeigt
    \item Der \Gls{Nutzer} navigiert zur Workflowübersicht und bekommt alle Workflows angezeigt
    \item Der \gls{Nutzer} klickt auf den \enquote{Neuen Workflow erstellen}-Knopf und bekommt den 
\end{itemize}

%
\subsection*{Workflow überprüfen}
\begin{itemize}
    \item Ein \Gls{Reviewer} meldet sich an und bekommt die Startseite angezeigt
    \item Der \Gls{Reviewer} navigiert zur Workflowübersicht und bekommt alle Workflows angezeigt
    \item Der \Gls{Reviewer} wählt einen laufenden Workflows aus und bekommt den Status des Workflows und ausgewählte Kennwerte angezeigt
    \item Der \gls{Reviewer} meldet sich ab
\end{itemize}

\subsection*{Workflow exportieren und importieren}
\begin{itemize}
    \item Ein \gls{Nutzer} meldet sich an und bekommt die Startseite angezeigt
    \item Der \gls{Nutzer} navigiert zur Workflowübersicht und bekommt alle Workflows angezeigt
    \item Der \gls{Nutzer} wählt eines der Workflows an und klickt auf den \enquote{Exportieren}-Knopf und eine .yml oder .json-Datei wird heruntergeladen
    \item Der \gls{Nutzer} 
\end{itemize}


\subsection*{Ergebnisse eines Workflows einsehen}
\begin{itemize}
    \item Ein \gls{Nutzer} meldet sich und bekommt die Startseite angezeigt
    \item Der \gls{Nutzer} navigiert zur Workflowübersicht und bekommt alle Workflows angezeigt
    \item Der \gls{Nutzer} wählt einen laufenden Workflows aus und bekommt eine Übersicht des ausgewählten Workflows angezeigt
    \item Der \gls{Nutzer} lädt sich die Ergebnisse des Workflows herunter
    \item Der \gls{Nutzer} meldet sich ab
\end{itemize}



\subsection*{Registrierung}
\begin{itemize}
    \item Ein neuer Benutzer ruft die Weboberfläche auf und bekommt die Login-Seite angezeigt
    \item Der neue Benutzer navigiert zur Registrierung
    \item Der neue Benutzer füllt das Registrierungsformular mit seinen Daten aus und gibt sein Passwort ein
    \item Der neue Benutzer schickt das Formular ab und gelangt zurück auf die Login-Seite
    \item Der neue Benutzer meldet sich mit seinem neuem \Gls{Reviewer}konto an
\end{itemize}

\subsection*{Nutzerverwaltung}
\begin{itemize}
    \item Ein \Gls{Admin} meldet sich an und bekommt die Startseite angezeigt
    \item Der \Gls{Admin} navigiert zur Benutzerverwaltung
    \item Der \Gls{Admin} bekommt alle Nutzerkonten und ihre Rollen angezeigt
    \item Der \Gls{Admin} wählt ein Konto aus und ändert seine Rolle von \Gls{Reviewer} zu \Gls{Admin}
    \item Der \Gls{Admin} wählt ein anderes Konto aus und setzt sein Password zurück
    \item Der \Gls{Admin} meldet sich ab
\end{itemize}


\subsection*{Hinzufügen von Computation Targets}
\begin{itemize}
    \item Ein \Gls{Admin} meldet sich an und ruft die Startseite auf
    \item Der \Gls{Admin} navigiert zur Serververwaltung
    \item Der \gls{Admin} erhöht den verfügbaren Hauptspeicher für die Workflows
    \item Der \gls{Admin} meldet sich ab
\end{itemize}
