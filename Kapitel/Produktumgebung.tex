\chapter{Produktumgebung}
Das Produkt ist als Webanwendung nach \gls{Client-Server} Modell konzipiert.

\section{Software}
\textbf{Server:}
Aufgeteilt in Ausführungsserver und Datenbank. Für beide gilt:
\begin{itemize} %Hier sollte noch mehr
    \setlength\itemsep{-1em}
    \item Ausführbar als Docker-Container
    \item Ausführbar über Nomad von HashiCorp
%Linux Version
%Andere Versionen für den Server
\end{itemize}
\textbf{Client:} Die Webanwendung ist von Linux und Windows aus erreichbar. 
Um die Anwendung auszuführen, ist eine Verbindung zum Universitätsnetzwerk erforderlich.
Die Webanwendung ist für folgende Internetbrowser optimiert:
\begin{itemize}
    \setlength\itemsep{-1em}
    \item Google Chrome (ab Version 96)
    \item Mozilla Firefox (ab Version 94) %.0.1
\end{itemize}
\newpage
\section{Hardware}
\textbf{Server}
\begin{itemize}
    \setlength\itemsep{-1em}
    \item 8 Kern CPU
    \item 8 GB Arbeitsspeicher
    \item 1 TB Hauptspeicher
\end{itemize}

%Welche Hardwarespezifikationen soll der Server haben? CPU (Kerne, Takt), Arbeitsspeicher und Hauptspeicher?
\textbf{Client:}
Außer einer Verbindung zum Universitätsnetzwerk stellt das Produkt keine besonderen Hardware-Anforderungen an den Client.

\section{Schnittstellen}
\renewcommand{\labelenumi}{/S\arabic{enumi}0/}
Das Produkt besitzt Schnittstellen zu folgenden Anwendungen
\begin{enumerate}
    \setlength\itemsep{-1em}
    \item \gls{M++}
    \item \gls{ABAQUS}
    \item \gls{SQL}: MySQL
    \item \gls{Apache Airflow}
\end{enumerate}
