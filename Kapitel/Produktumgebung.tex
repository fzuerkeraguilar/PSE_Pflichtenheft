\chapter{Produktumgebung}
Das Produkt ist im \gls{Client-Server} Prinzip konzipiert.

\section{Software}
\textbf{Server}
\begin{itemize}
    \item Läuft auf \gls{Virtual Machine} Server
    \item Ausführbar auf Linux
%Linux Version
%Andere Versionen für den Server
\end{itemize}
\textbf{Client} Die Webanwendung ist von allen Betriebssystemen aus erreichbar. Die Anwendung ist mit Chrome und Firefox benutzbar. Nur eine Verbindung zum Universitätsnetzwerk ist erforderlich um die Anwendung auszuführen. \\[0.1in] Konkret unterstützen wir:
\begin{itemize}
    \setlength\itemsep{-1em}
    \item Chrome ab Version 96
    \item Firefox ab Version 94 %.0.1
\end{itemize}
\section{Hardware}
\textbf{Server}
\begin{itemize}
    \item 8 Kern CPU
    \item 16 GB Arbeitsspeicher
    \item 2 TB Hauptspeicher
\end{itemize}

%Welche Hardwarespezifikationen soll der Server haben? CPU (Kerne, Takt), Arbeitsspeicher und Hauptspeicher?
\textbf{Client}
Außer einer Verbindung zum Universitätsnetzwerk wird keine Hardware zur Verwendung der Anwendung benötigt. 

\section{Schnittstellen}
\renewcommand{\labelenumi}{/S\arabic{enumi}0/}
Das Produkt besitzt eine Schnittstelle zu
\begin{enumerate}
    \setlength\itemsep{-1em}
    \item \gls{M++}
    \item \gls{SQL}: MySQL
    \item Airflow Scheduler
\end{enumerate}
