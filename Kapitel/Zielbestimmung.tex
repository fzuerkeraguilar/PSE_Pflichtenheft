\chapter{Zielbestimmung}
Durch das Produkt sollen Materialwissenschaftler in die Lage versetzt werden, den Prozess des Kompilierens und Ausführens von Simulationen effizienter durchzuführen. 
Zusätzlich sollen die Ergebnisse der Simulationen über das Produkt einsehbar sein. \newline
\section{Musskriterien}
Folgende Funktionen müssen erfüllt werden:
\renewcommand{\labelenumi}{/M\arabic{enumi}0/}
\begin{enumerate}
    \setlength\itemsep{-1em}
    \item Workflows verwalten
    \item Workflows entwerfen %als Graph
    \item Workflows ausführen
    \item Ergebnisse anzeigen
    %\item Konfiguration der Workflowschritte durch Dialog
    %\item Graphische Nutzeroberfläche als Webanwendung
    \item Qualitätskontrolle
    %\item Fehlschläge berichten
    %\item Exportieren von Workflows
    %\item Importieren von Workflows
    %\item Details der Ausführung herunterladbar
    %\item Zustand laufender Workflows einsehen
    \item Nutzer verwalten
    \item Konfiguration ändern
    %\item Nutzerrechte verwalten
    %\item Nutzerdaten löschen
\end{enumerate}
\newpage
\section{Wunschkriterien}
Folgende Funktionen wären wünschenswert:
\renewcommand{\labelenumi}{/W\arabic{enumi}0/}
\begin{enumerate}
    \setlength\itemsep{-1em}
    \item Schutz vor SQL Injektionen
    \item Sauberes Beenden der Anwendung
    \item Validierung der Custom-Schritte vor der Ausführung
    \item Überwachung der Datenqualität
    \item Ändern von plotskripten ohne die Simulation erneut ausführen zu müssen
\end{enumerate}

\section{Abgrenzungskriterien}
Folgende Funktionen sind kein Ziel des Projektes:
\renewcommand{\labelenumi}{/A\arabic{enumi}0/}
\begin{enumerate}
    \setlength\itemsep{-1em}
    \item Einsatz nicht auf Handhelds, Palms oder anderen mobilen Rechner eingeplant 
\end{enumerate}
\renewcommand{\labelenumi}{\arabic{enumi}}