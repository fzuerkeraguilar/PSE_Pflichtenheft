\chapter{Entwicklungsumgebung}
Das fertige Programm basiert auf \gls{Apache Airflow} und wird um die gewünschten Funktionalitäten erweitert.
\section{Programmiersprache}
Das Produkt wird mit folgenden Programmiersprachen entwickelt:
\begin{itemize}
    \setlength\itemsep{-1em}
    \item Python mit Flask
    \item Typescript
\end{itemize}
Wo benötigt verwenden wir zudem:
\begin{itemize}
    \setlength\itemsep{-1em}
    \item Bash-Shellskript 
    \item HTML
    \item CSS
\end{itemize}
\newpage
\section{Software}
Folgende Software unterstützt die Entwicklung:
%Zur Entwicklung nutzen wir folgende Software:
\begin{itemize}
    \setlength\itemsep{-1em}
    \item Apache Airflow zum Ausführen der \glspl{Workflow}
    \item git mit GitLab zur Versionsverwaltung
    \item MySQL als Datenbank
    \item Docker mit Normad für Deployment
    \item black und pyupgrade zur Stilprüfung und zur statischen Code-Analyse
\end{itemize}


\section{Hardware}
Die Entwicklung erfolgt auf Standard-PCs. An diese werden außer einer Internetverbindung keine speziellen Hardwareanforderungen gestellt.
%Zur Entwicklung wird außer einer Internetverbindung keine zusätzliche Hardware benötigt.