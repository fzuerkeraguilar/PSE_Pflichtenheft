\chapter{Produktfunktionen}

\paragraph{/M10/ \textit{\glspl{Workflow} verwalten} wird umgesetzt durch:}

\newcounter{FAs}

\renewcommand{\labelenumi}{/FA\arabic{enumi}0/}
\begin{enumerate}
    \setlength\itemsep{-1em}    
    % Muss wo anders \item Darstellung laufender \glspl{Workflow}
    \item Darstellung abgeschlossener \glspl{Workflow}
    \item Information von bereits abgeschlossenen \glspl{Workflow} einsehen
    \item Herunterladen ausgewählter Informationen von abgeschlossenen \glspl{Workflow}
    \item Darstellung abgeschlossener \glspl{Workflow}
    \item \gls{Workflow} importieren
    \item \gls{Workflow} exportieren
    \setcounter{FAs}{\value{enumi}}
\end{enumerate}

\paragraph{/M20/ \textit{\glspl{Workflow Schablone} verwalten} wird umgesetzt durch:}
\begin{enumerate}
    \setlength\itemsep{-1em}
    \setcounter{enumi}{\value{FAs}}
    \item Darstellung aller \glspl{Workflow Schablone}
    \item Darstellung der \gls{Workflow Schablone} Schritte
    \item \gls{Workflow Schablone} löschen
    \item \gls{Workflow Schablone} umbenennen
        \item \gls{Workflow Schablone} importieren
    \item \gls{Workflow Schablone} exportieren
    \setcounter{FAs}{\value{enumi}}
\end{enumerate}

\paragraph{/M30/ \textit{Parametrierung von \glspl{Workflow Schablone}} wird umgesetzt durch:}
\begin{enumerate}
    \setlength\itemsep{-1em}
    \setcounter{enumi}{\value{FAs}}
    \item Hochladen von Config-Dateien
    \item Bearbeiten von Config-Dateien als Rohtext
    %\item Bearbeiten von Config-Dateine als \gls{Key-Value-Pair}-Liste
    \setcounter{FAs}{\value{enumi}}
\end{enumerate}

\paragraph{/M40/ \textit{\glspl{Workflow Schablone} entwerfen} wird umgesetzt durch:}


\renewcommand{\labelenumi}{/FA\arabic{enumi}0/}
\renewcommand{\labelenumii}{/FA\arabic{enumi}\arabic{enumii}/}
\begin{enumerate}
    \setcounter{enumi}{\value{FAs}}
    \setlength\itemsep{-1em}
    \item Programmbasierte Erstellung von \glspl{Workflow Schablone}
    \vspace{-5mm}
    \begin{enumerate}
        \setlength\itemsep{-1em}
        \item Erstellung von Standard Machine Learning \glspl{Workflow Schablone} %Ist das so gut?
        \item Erstellung von Theory-guided Feature \glspl{Workflow Schablone} %Ist das so gut?
        \item Erstellung von Theory-guided Intermediate Value \glspl{Workflow Schablone} %Ist das so gut?
        \item Erstellung von Theory-guided Hybrid Model \glspl{Workflow Schablone} %Ist das so gut?
    \end{enumerate}
    \renewcommand{\labelenumi}{/FA\arabic{enumi}0/}
    \item Hinzufügen von \glspl{Workflow Schablone}-schritten
    \item Entfernen von \glspl{Workflow Schablone}-schritten
    %\item Auswählen von Workflowschritten
    %\item Konfiguration der Schritte durch Dialog (Ist das hier das gleiche wie Konfiguration der Config-Dateien?)
    \item Hochladen von \gls{Python}code% und Festlegung der Parameter für Nutzer 
    %\item Festlegung der Parameter für Nutzer
    \setcounter{FAs}{\value{enumi}}
\end{enumerate}

%Vielleicht: Workflow Ausführung
\paragraph{/M50/ \textit{\glspl{Workflow} Ausführen} wird umgesetzt durch:}

\begin{enumerate}
    \setlength\itemsep{-1em}
    \setcounter{enumi}{\value{FAs}}
    %\item Workflow aus Liste auswählen
    %\item Parameter für Durchführung angeben
    \item Ausführung starten
    \item Ausführung abbrechen
    \item Speichern von Metadaten eines \glspl{Workflow}
    \item Informationen von laufenden Workflows einsehen
    \item Herunterladen ausgewählter Informationen von laufenden Workflows
    %\item Bearbeiten von Config-Dateien
    \setcounter{FAs}{\value{enumi}}
\end{enumerate}

\paragraph{/M60/ \textit{\gls{Workflow}-Ergebnisse anzeigen} wird umgesetzt durch:}
\begin{enumerate}
    \setlength\itemsep{-1em}
    \setcounter{enumi}{\value{FAs}}
    \item Anzeigen der Ergebnisse der \gls{Workflow} als Rohtext
    \item Anzeigen der Metadaten der \gls{Workflow}
    \item Anzeigen des Logs der \gls{Workflow}
    \item eigene Visualisierungsscripte, die vor der Ausführung festgelegt werden
    \setcounter{FAs}{\value{enumi}}
\end{enumerate}

\paragraph{/M70/ \textit{Workflow-Instanz Ergebnisse exportieren} wird umgesetzt durch:}
\begin{enumerate}
    \setlength\itemsep{-1em}
    \setcounter{enumi}{\value{FAs}}
    \item Auswahl von Bestandteilen aller erzeugten Ergebnisse
    \item Auswahl von erzeugten Metadaten
    \item Herunterladen als Archivdatei
    \setcounter{FAs}{\value{enumi}}
\end{enumerate}


\paragraph{/M80/ \textit{Qualitätskontrolle} wird umgesetzt durch:}

\begin{enumerate}
    \setlength\itemsep{-1em}
    \setcounter{enumi}{\value{FAs}}
    \item Systemressourcen überwachen
    \item Datenbankstatus überwachen
    \item Benachrichtigung bei Fehlschlägen über eine Nachricht über E-Mail
    \setcounter{FAs}{\value{enumi}}
\end{enumerate}
    
\paragraph{/M90/ \textit{Nutzer verwalten} wird umgesetzt durch:}

\begin{enumerate}
    \setlength\itemsep{-1em}
    \setcounter{enumi}{\value{FAs}}
    \item Login über Webseite
    \item neue Benutzer registrieren
    \item Unterscheidung von Nutzern in \gls{Reviewer}, \gls{Nutzer} und \gls{Admin}
    \item Passwort zurücksetzen
    \item Änderung der Nutzerrechte durch \glspl{Admin}
    %\item Löschung aller Nutzerdaten durch \glspl{Admin}
    \item Löschen von Nutzern
    \item Sperren von Nutzern %braucht man das?
    \setcounter{FAs}{\value{enumi}}
\end{enumerate}
    
\paragraph{/M100/ \textit{Anwendungskonfiguration ändern} wird umgesetzt durch:}

\begin{enumerate}
    \setlength\itemsep{-1em}
    \setcounter{enumi}{\value{FAs}}
    \item Konfiguration der Datenbank
    \item Einstellung von \gls{Limits} %was für Limits?
    \item Erkennen verfügbarer \gls{Ressourcen}
    \item Festlegen verfügbarer Ressourcen
\end{enumerate}